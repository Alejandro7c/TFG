\capitulo{1}{Introducción}

Desde el inicio del estudio de ingenierías, siempre ha resultado más sencilla la comprensión por parte de los alumnos de conceptos teóricos si estos son vistos en la práctica. Sin embargo, no siempre es posible tener acceso a estas facilidades debido a la no inclusión de un software dirigido al estudiante, entre otras desventajas.

Hasta el momento los alumnos de la Universidad de Burgos han tenido que trabajar con las placas NXP FRDM-K64F de una manera arcaica, con un software de terceros, a través de una consola de comandos que enviaba mensajes directamente a la placa, sin poder optar a una visualización apropiada de los datos.

Partiendo de este principio y teniendo en cuenta los últimas tecnologías implantados en la Universidad de Burgos, el objetivo de este proyecto es el desarrollo de una interfaz hombre-máquina capaz de comunicarse con una placa NXP FRDM-K64F en un entorno amigable, permitiendo al alumno una mayor libertad para trabajar con dicho dispositivo.

\subsection{1.1. Estructura de la memoria}

La memoria sigue la siguiente estructura:
\begin{itemize}
	\item \textbf{Introducción:} breve descripción del problema a resolver y la solución propuesta. Estructura de la memoria y listado de materiales adjuntos.
	\item \textbf{Objetivos del proyecto:} exposición de los objetivos que persigue el proyecto.
	\item \textbf{Conceptos teóricos:} en este apartado se describen los conceptos que se necesitan saber antes de poder comprender todo el proyecto.
	\item \textbf{Técnicas y herramientas:} donde se indican todas las técnicas, bibliotecas, lenguajes, y otras herramientas que se han utilizado durante el proyecto.
	\item \textbf{Aspectos relevantes del desarrollo:} descripción del desarrollo general que ha llevado el proyecto.
	\item \textbf{Trabajos relacionados:} se nombran otros trabajos similares o relacionados con este proyecto. 
	\item \textbf{Conclusiones y líneas de trabajo futuras:} conclusiones obtenidas tras la realización del proyecto y posibilidades de mejora o expansión de la solución aportada.
\end{itemize}

Junto con la memoria se proporcionan los siguientes anexos:
\begin{itemize}
	\item \textbf{Plan del proyecto software:} planificación temporal y estudio de viabilidad del proyecto.
	\item \textbf{Especificación de requisitos del software:} se describe la fase de análisis; los objetivos generales, el catálogo de requisitos del sistema y la especificación de requisitos funcionales y no funcionales.
	\item \textbf{Especificación de diseño:} se describe la fase de diseño; el ámbito
del software, el diseño de datos, el diseño procedimental y el diseño arquitectónico.
	\item \textbf{Manual del programador:} recoge los aspectos más relevantes relacionados
con el código fuente (estructura, compilación, instalación, ejecución, pruebas, etc.).
	\item \textbf{Manual de usuario:} guía de usuario para el correcto manejo de la
aplicación.
\end{itemize}


\subsection{1.2. Materiales adjuntos}
\begin{itemize}
	\item \textbf{Introducción:} 
	\item \textbf{Introducción:} 
	\item \textbf{Introducción:} 
	\item \textbf{Introducción:} 
\end{itemize}