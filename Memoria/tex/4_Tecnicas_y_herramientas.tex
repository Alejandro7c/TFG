\capitulo{4}{Técnicas y herramientas}

En este apartado se detallarán las técnicas y herramientas utilizadas para desarrollar el proyecto.

\subsection{4.1. Metodología}

\subsubsection{4.1.1. Scrum}

\href{https://proyectosagiles.org/que-es-scrum/}{Scrum} es un marco de trabajo para el desarrollo de \textit{software} que se engloba dentro de las metodologías ágiles. Sus principales características son la adopción de una estrategia de desarrollo incremental, basar la calidad del resultado en el conocimiento tácito de los integrantes de equipos auto organizados, y en el solapamiento de las diferentes fases del desarrollo \cite{wiki:scrum}.

En este caso los \textit{sprints} han sido de dos semanas, tras los cuales se realizaba una reunión para definir las nuevas tareas del siguiente \textit{sprint} y realizar una retroalimentación del \textit{sprint} recién hecho.

\subsubsection{4.1.2. Técnica de Pomodoro}

Para mejorar la concentración durante el desarrollo del proyecto se ha utilizado la técnica Pomodoro, consistente en intervalos de 25 minutos de concentración intensa seguidos de descansos de 5 minutos, y descansos de 30 minutos después de haber completado cuatro ciclos 25+5 \cite{wiki:pomodoro}.

\subsection{4.2. Patrones de diseño}

\subsubsection{4.2.1. MVC - Modelo Vista Controlador}

A la hora de estructurar el desarrollo de la aplicación se ha optado por la utilización del Modelo Vista Controlador, consiguiendo así separar la capa que representa la realidad, la capa que conoce los métodos y atributos del modelo, recibiendo y realizando las peticiones del usuario, y la capa visible para el usuario \cite{web:patronDis}.

\subsection{4.3. Control de versiones}

\subsubsection{4.3.1. Git}

\href{https://git-scm.com/}{Git} es, a día de hoy, el sistema de control de versiones distribuido más usado del mundo. Su propósito es llevar el registro de los cambios realizados en los archivos de un ordenador y coordinar el trabajo de un equipo formado por varias personas sobre un mismo archivo \cite{wiki:Git}.

La otra opción tenida en cuenta para este proyecto fue \href{https://subversion.apache.org/}{Subversion}, que fue descartada por la facilidad de uso que ofrece Git, la familiaridad con el sistema de control de versiones y las herramientas con las que se puede complementar.

\subsection{4.4. Hosting de repositorio}

\subsubsection{4.4.1. GitHub}

\href{https://github.com/}{GitHub} es una plataforma de desarrollo colaborativo para alojar proyectos utilizando el sistema de control de versiones Git.\cite{wiki:github}. Ofrece un repositorio Git donde los desarrolladores pueden almacenar, compartir, testear y colaborar en proyectos web.

Al comenzar el proyecto también se tuvo en cuenta la posibilidad de utilizar \href{https://about.gitlab.com/}{GitLab} pero fue descartado por la compatibilidad de GitHub con otras herramientas usadas en el desarrollo del proyecto y por la familiaridad con dicho hosting de repositorio.

\subsection{4.5. Gestión del proyecto }

\subsubsection{4.5.1. ZenHub}

\href{https://www.zenhub.com/}{ZenHub} es una herramienta gratuita para proyectos pequeños u \textit{open source} cuya finalidad es la gestión de proyectos en GitHub. Esta herramienta trabaja de manera totalmente integrada ofreciendo un tablero canvas en el que cada \textit{issue} nativo de GitHub se representa como una tarea. A estas tareas se le pueden asignar prioridad, dependencias, estimaciones de tiempo de realización, colaboradores y el \textit{sprint} al que pertenece. ZenHub también permite graficar los avances llevados a cabo en el proyecto.

\imagen{zenhubOverview}{ZenHub Overview}

En un principio también se barajó utilizar GitHub Proyect pero tras ver los vídeos explicativos subidos en la plataforma de la universidad me decanté por ZenHub.

\subsection{4.6. Entorno de desarrollo integrado (IDE)}

\subsubsection{4.6.1. Microsoft Visual Studio 2017}

\href{https://visualstudio.microsoft.com/es/vs/}{Microsoft Visual Studio 2017} es un entorno de desarrollo integrado para sistemas operativos Windows. Soporta múltiples lenguajes de programación, tales como F\#, Python, C++, C\#, Visual Basic .NET, Java, Ruby y PHP, al igual que entornos de desarrollo web, como ASP.NET MVC, Django, etc., a lo cual hay que sumarle las nuevas capacidades online bajo Windows Azure en forma del editor Monaco \cite{wiki:visualstudio}.

Cuando se inició el proyecto también se tuvo en cuenta usar versiones anteriores de Visual Studio por sus compatibilidades a la hora de desarrollar Windows Forms sobre C++, pero tras enfrentar las ventajas de cada IDE se decidió usar el Visual Studio 2017 con Windows Forms en C\#.

\subsubsection{4.6.2. Texmaker}

\href{http://www.xm1math.net/texmaker/}{Texmaker} es un editor multiplataforma gratuito (licencia GNU GLP v3.0) para \LaTeX que integra la mayoría de herramientas necesarias para la escritura de documentos, corrector ortográfico, auto-completado, resultado de análisis, visor de PDF integrado, entre otros.

En un principio se trató de utilizar \href{https://code.visualstudio.com/}{Visual Studio Code} con la extensión LaTeX Workshop pero no ofreció un uso eficiente a la hora de compilar toda la solución \LaTeX.

\subsection{4.7. Comunicación}

La comunicación con los tutores se ha realizado a través de varias vías:
\begin{itemize}
	\item Correo electrónico.
	\item Reuniones presenciales. 
	\item Canvas de ZenHub, donde se puede comentar cada tarea.
\end{itemize}

También se pensó en utilizar herramientas como \href{https://slack.com/intl/es-es/}{Slack} pero se descartaron porque se vieron innecesarias.

\subsection{4.8. Documentación}

\subsubsection{4.8.1. LaTeX}

\LaTeX es un sistema de composición de textos, orientado a la creación de documentos escritos que presenten una alta calidad tipográfica. Por sus posibilidades y características, es usado de forma especialmente intensa en la generación de libros científicos y artículos que incluyen, entre otros elementos, expresiones matemáticas \cite{wiki:latex}.

Para la correcta presentación de esta documentación que está siendo leída se ha utilizado la plantilla ofrecida por la Universidad de Burgos, con la finalidad de alcanzar una mayor afinidad en el formato de la misma.

En un principio se planteó la posibilidad de utilizar tanto Open Office Writter como Microsoft Word, en el primer caso porque la Universidad de Burgos ofrecía una plantilla oficial, y en el segundo porque la universidad incluye una licencia para su uso, sin olvidar lo familiarizado que estoy a ella.

\subsection{4.9. Test}

\subsubsection{4.9.1. Codacy}

Para llevar a cabo una revisión del código de la aplicación se utilizó \href{https://www.codacy.com/}{Codacy}, herramienta que ejecuta de manera automática exámenes sobre el código de cada \textit{commit} que se realice en Git. Codacy informa de los problemas de seguridad, de cobertura, de duplicidad y de complicidad que pueda tener el código, catalogando a este según los baremos de la propia aplicación y graficando los resultados.

\imagen{codacyDashboard}{Codacy Dashboard}

\subsection{4.10. Otras herramientas}

\subsubsection{4.10.1. Paint.NET}

\href{https://www.getpaint.net/index.html}{Paint.NET} es una herramienta de edición de imágenes sobre entorno de Windows. Ofrece una gran cantidad de posibilidades a la hora de editar una imagen, es un programa ligero y de muy fácil utilización. También se tuvo en cuenta \href{http://www.gimp.org.es/}{Gimp} como herramienta de edición de imágenes pero se descartó porque tardaba mucho en ejecutarse.