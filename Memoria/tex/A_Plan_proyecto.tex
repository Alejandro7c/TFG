\apendice{Plan de Proyecto Software}

\section{Introducción}

A lo largo de este apéndice se tratará todo aquello relacionado con la planificación del proyecto, considerándose esta un punto clave para cualquier desarrollo de \textit{software}. En esta fase se estima tanto el dinero, como el trabajo y el tiempo que se va a emplear en completar el proyecto a través de un análisis minucioso de los recursos necesarios.

La fase de planificación se encuentra dividida en:

\begin{itemize}
	\item Planificación temporal.
	\item Estudio de viabilidad.
	\begin{itemize}
		\item Viabilidad económica.
		\item Viabilidad legal.
	\end{itemize}
\end{itemize}

\section{Planificación temporal}

En esta sección se elaborará un programa de tiempos en los que se estima la duración de cada una de las partes del proyecto. Partiendo del establecimiento de una fecha de inicio y una fecha de finalización estimada, a través tanto del peso de cada una de las tareas como de los requisitos necesarios para poder empezar cada una.

Al inicio del proyecto se decidió utilizar \textit{Scrum} como metodología ágil para la gestión del proyecto. Debido a que el equipo formado no contaba con más de cuatro personas, no se ha podido seguir a rajatabla, pero sí que se han seguido las líneas generales de esta filosofía:

\begin{itemize}
	\item Estrategia de desarrollo incremental a través de \textit{sprints} (o iteraciones) y revisiones.
	\item La duración media de cada \textit{sprint} era de dos semanas.
	\item Al finalizar cada \textit{sprint} se realizaban reuniones en las que se revisaba el incremento en el proyecto y se planificaba el siguiente \textit{sprint}.
	\item Tras la planificación del \textit{sprint} se creaban una serie de tareas a realizar, las cuales eran estimadas y priorizadas en un tablero \textit{canvas}.
	\item La monitorización del progreso del proyecto se llevó a cabo a través de gráficos \textit{burndown}.
\end{itemize}

Cabe comentar que la estimación se realizó a través de \textit{story points} (incluidos por ZenHub) que tienen una traducción temporal mostrada en la Tabla A.1.

\tablaSmall{Equivalencia entre \textit{story points} y tiempo.}{l l}{equivalenciaEstimacion}{Story points & Estimación temporal \\}{
1 & 30'\\
2 & 2h\\
3 & 3h\\
5 & 5h\\
8 &10h\\
13 & 18h\\
21 & 26h\\
40 & 50h\\
}

\textit{Sprints} llevados a cabo:

\section{Estudio de viabilidad}

\subsection{Viabilidad económica}
Viabilidad economica: donde se estiman los costes y los benecios
que puede suponer la realizacion del proyecto.

\subsection{Viabilidad legal}
Viabilidad legal: el contexto en el que se ejecuta el proyecto esta regulado
por una serie de leyes. Se deben analizar todas aquellas que afecten
al proyecto. En el caso del software, las licencias y la Ley de Proteccion
de Datos pueden ser los temas mas relevantes.

