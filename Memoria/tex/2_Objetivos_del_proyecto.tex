\capitulo{2}{Objetivos del proyecto}

A continuación se detallan los diversos objetivos que han motivado la realización del proyecto.

\section{Objetivos generales}

\begin{itemize}
	\item Crear una interfaz hombre-máquina amigable que permita la monitorización y modificación de los datos gestionados por una placa NXP.
	\item Ayudar a los nuevos estudiantes de la Universidad de Burgos en su comprensión del funcionamiento de una una planta piloto y los sistemas de control necesarios para controlarla. Fundamentalmente orientado para las prácticas de las asignaturas de control en el grado de Electrónica Industrial y Automática, el grado de ingeniería Informática y el máster de Ingeniería Industrial.
	\item Aportar información extra a los datos recibidos por la placa que ayude a la comprensión de los cambios experimentados en la planta piloto.
	\item Dar la posibilidad de almacenar los datos recibidos de la planta piloto de una manera concisa, estructurada y de fácil acceso.
\end{itemize}

\section{Objetivos técnicos}

\begin{itemize}
	\item Desarrollar una aplicación en .NET Framework - Windows Forms para entornos Windows.
	\item Utilizar Git como herramienta de control de versiones distribuido junto con GitHub.
	\item Aplicar la metodología ágil Scrum en el desarrollo del software.
	\item Utilizar la herramienta ZenHub como herramienta de gestión de proyectos.
	\item Documentar a través de la herramienta \LaTeX.
	\item Implementar la comunicación serie entre la aplicación y la la placa para conseguir una comunicación online con la misma y el proceso
	\item Distribuir la aplicación resultante para entornos Windows.
\end{itemize}

\section{Objetivos personales}

\begin{itemize}
	\item Abarcar el máximo número de conocimientos vistos en el grado, como por ejemplo programación orientada a objetos, bases de datos, ingeniería del software, comunicaciones, etc.
	\item Realizar una aportación a la modernización de los recursos de la Universidad de Burgos.
	\item Profundizar en el desarrollo de aplicaciones .NET y en la utilización del entorno de desarrollo Visual Studio.
	\item Trabajar con placas Kinetis de NXP.
	\item Explorar herramientas y metodologías de vanguardia en el mercado laboral.
\end{itemize}