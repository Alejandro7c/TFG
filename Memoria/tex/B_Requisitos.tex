\apendice{Especificación de Requisitos}

\section{Introducción}

Este anexo recoge la especificación de requisitos que define el comportamiento del sistema desarrollado. De esta manera, se consigue tener la documentación correspondiente al análisis de la aplicación y un documento contractual entre el cliente y el equipo de desarrollo.

Para llevar a cabo esta tarea se han seguido las recomendaciones del estándar IEEE 830-1998 sobre una buena especificación de requisitos \cite{web:ieee830}:
\begin{itemize}
	\item Consistente.
	\item Completa.
	\item inequívoca.
	\item Trazable.
	\item Correcta.
	\item Modificable.
	\item Priorizable.
	\item Verificable.
\end{itemize}

\section{Objetivos generales}

\begin{itemize}
	\item Crear una interfaz hombre-máquina amigable que permita la monitorización y modificación de los datos gestionados por una placa NXP.
	\item Ayudar a los nuevos estudiantes de la Universidad de Burgos en su comprensión del funcionamiento de una una planta piloto y los sistemas de control necesarios para controlarla. Fundamentalmente orientado para las prácticas de las asignaturas de control en el grado de Electrónica Industrial y Automática, el grado de ingeniería Informática y el máster de Ingeniería Industrial.
	\item Aportar información extra a los datos recibidos por la placa que ayude a la comprensión de los cambios experimentados en la planta piloto.
	\item Dar la posibilidad de almacenar los datos recibidos de la planta piloto de una manera concisa, estructurada y de fácil acceso.
\end{itemize}

\section{Catalogo de requisitos}

\subsection{Requisitos funcionales}

\begin{itemize}
	\item \textbf{RF-1} El usuario debe ser capaz de trabajar con distintos proyectos en la aplicación.
	\begin{itemize}
		\item \textbf{RF-1.1} El usuario debe ser capaz de crear un nuevo proyecto desde la aplicación.
		\item \textbf{RF-1.2} El usuario debe ser capaz de cargar un proyecto existente desde la aplicación.
		\item \textbf{RF-1.3} El usuario debe ser capaz de modificar un proyecto cargado en la aplicación.	
	\end{itemize}
	\item \textbf{RF-2} El usuario debe ser capaz de cambiar de idioma durante ejecución de la aplicación.
	\item \textbf{RF-3} El usuario debe ser capaz de acceder a la ayuda de la aplicación desde cualquier ventana.
	\item \textbf{RF-4} El usuario debe ser capaz de modificar el estado de la conexión con el puerto serie.
	\begin{itemize}
		\item \textbf{RF-4.1} El usuario debe ser capaz de abrir una comunicación con el puerto serie desde la aplicación.
		\item \textbf{RF-4.2} El usuario debe ser capaz de cerrar una comunicación con el puerto serie desde la aplicación.
	\end{itemize}
	\item \textbf{RF-5} El usuario debe ser capaz de modificar los valores de las variables de escritura que tenga el proyecto.
	\item \textbf{RF-6} El usuario debe ser capaz de acceder a los datos devueltos por la placa NXP y almacenados en la base de datos.
	\begin{itemize}
		\item \textbf{RF-6.1} El usuario debe ser capaz de visualizar los datos almacenados en la base de datos en una gráfica.
		\item \textbf{RF-6.2} El usuario debe ser capaz de visualizar los datos almacenados en la base de datos en una nueva ventana.
		\item \textbf{RF-6.3} El usuario debe ser capaz de guardar los datos almacenados en la base de datos en un archivo de texto.
	\end{itemize}
	
\end{itemize}

\subsection{Requisitos no funcionales}

\begin{itemize}
	\item \textbf{RNF-1} La interfaz debe ser sencilla e intuitiva.
	\item \textbf{RNF-2} La aplicación debe gestionar las posibles excepciones que acontezcan durante la ejecución de la misma, mostrando al usuario las que sean pertinentes.
	\item \textbf{RNF-3} La aplicación debe ser mantenible, facilitando la escalabilidad a la hora de añadir características.
	\item \textbf{RNF-4} La aplicación debe ser estar preparada para soportar varios idiomas.
	\item \textbf{RNF-5} La aplicación debe poder ser desplegable con facilidad y en poco tiempo.
	
\end{itemize}

\section{Especificación de requisitos}

En esta sección se desarrollarán los distintos casos de uso, los cuales cubren los requisitos funcionales presentados en la sección anterior.

\tablaSinColores{CU-01 Crear proyecto}{l l}{2}{cu01}{\textbf{CU-01} & \textbf{Crear proyecto} \\}{
\textbf{Autor} & Francisco Crespo Diez \\
\textbf{Requisitos asociados} & RF-1.1 \\
\textbf{Descripción} & Permite al usuario crear un proyecto y cargarlo \\ & en la aplicación. \\
\textbf{Precondiciones} & Ninguna\\
\textbf{Acciones} & El usuario rellena el formulario con los datos del \\ & proyecto. \\
\textbf{Postcondición} & El proyecto se carga en la aplicación. \\
 			& Si no existe, se crea la base de datos con el nombre\\
 			& TFG\_DB.\\
 			& Si no existe, se crea la tabla cuyo nombre será el\\
 			& nombre del proyecto. En caso de existir dicha tabla\\
 			& se sobrescribe.\\
\textbf{Excepciones} & Fallo al crear la base de datos o la tabla por falta\\
			& de archivos de configuración de la base de datos. \\
\textbf{Importancia} & Alta. \\
}

\tablaSinColores{CU-02 Cargar proyecto}{l l}{2}{cu02}{\textbf{CU-02} & \textbf{Cargar proyecto} \\}{
\textbf{Autor} & Francisco Crespo Diez \\
\textbf{Requisitos asociados} & RF-1.2 \\
\textbf{Descripción} & Permite al usuario cargar un proyecto ya creado \\ 
			& en la aplicación \\
\textbf{Precondiciones} & Ninguna\\
\textbf{Acciones} 	& El usuario selecciona el archivo .txt con la  \\ 
			& configuración del proyecto. \\
\textbf{Postcondición} & El proyecto se carga en la aplicación. \\
 			& Si no existe, se crea la base de datos con el nombre\\
 			& TFG\_DB.\\
 			& Si no existe, se crea la tabla cuyo nombre será el\\
 			& nombre del proyecto. En caso de existir dicha tabla\\
 			& se sobrescribe.\\
\textbf{Excepciones} & Fallo al crear la base de datos o la tabla por falta\\
			& de archivos de configuración de la base de datos. \\
\textbf{Importancia} & Alta \\
}

\tablaSinColores{CU-04 Modificación de proyecto cargado}{l l}{2}{cu04}{\textbf{CU-04} & \textbf{Modificar proyecto cargado}\\}{
\textbf{Autor} & Francisco Crespo Diez \\
\textbf{Requisitos} asociados & RF-1.4 \\
\textbf{Descripción} & Permite al usuario modificar un proyecto cargado \\ 
			& en la aplicación. \\
\textbf{Precondiciones} & Debe haber un proyecto cargado en la aplicación.\\
\textbf{Acciones} 	& El usuario selecciona el archivo .txt con la  \\ 
			& configuración del proyecto. \\
\textbf{Postcondición} & El proyecto se carga en la aplicación. \\
 			& Si no existe, se crea la base de datos con el nombre\\
 			& TFG\_DB.\\
 			& Si no existe, se crea la tabla cuyo nombre será el\\
 			& nombre del proyecto. En caso de existir dicha tabla\\
 			& se sobrescribe.\\
\textbf{Excepciones} & Fallo al crear la base de datos o la tabla por falta\\
			& de archivos de configuración de la base de datos. \\
\textbf{Importancia} & Alta \\
}
