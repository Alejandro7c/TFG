\capitulo{8}{Conclusiones y líneas de trabajo futuras}

En esta sección se exponen tanto las conclusiones derivadas del trabajo, como las posibles líneas de trabajo futuras con las que se puede dar continuidad al proyecto. 

\section{Conclusiones}

Desde un punto de vista general y de una manera personal, finalizo este Trabajo de Fin de Grado llevándome conmigo una sensación positiva, tanto por todo lo que he aprendido gracias a la resolución de los diferentes problemas que me han aparecido, como por dejar una herramienta útil  para la formación de futuros alumnos de la Universidad de Burgos. Poder haber participado en este proyecto comenzado por mis compañeros Rubén Zambrana Rodríguez y María Isabel Revilla Izquierdo me hace sentir que formo parte de una idea mayor que contiene una finalidad altruista y formativa.

Ciñéndome a todo lo relacionado a la documentación, resulta interesante ver como todo aquello aprendido durante la carrera sobre diagramas, previsiones y un largo etcétera de conceptos, es útil para el planteamiento de un proyecto, ayudando al alumno a ponerse en una situación real de desarrollo de software y en la que puede ver todo lo que está involucrado.

Si atiendo a la parte técnica, he tenido la suerte de poder desarrollar la aplicación en el entorno y lenguaje a los que más habituado estoy, por lo que gracias a ello he conseguido ampliar mis conocimientos sobre C\#, Visual Studio y SQL Server, preparándome un poco más para el mundo laboral. Cabe destacar que a la vez que realizaba este trabajo de fin de grado estaba acudiendo a prácticas extracurriculares en la empresa Keyland Sistemas de Gestión, de donde he obtenido muchas ideas y ayuda tanto para organizar el proyecto como para solventar los problemas que han ido aconteciendo.

En resumen, puedo concluir orgulloso que ha sido un proyecto con el que me he nutrido tanto personal como profesionalmente. Durante este tiempo he sido capaz de medirme y ver hasta dónde alcanzan tanto mis conocimientos como mi habilidad para plasmarlos en un entorno real.

\section{Líneas de trabajo futuras}

Durante la programación de la aplicación aparecieron posibles mejoras que se pueden añadir a la solución en líneas de trabajo futuras.

\subsection{Comunicación a través del puerto Ethernet}

Atendiendo a las posibilidades que ofrece la placa NXP FRDM-K64F resultaría interesante ampliar las posibilidades de comunicación a través  del puerto Ethernet. Esta línea de trabajo se ha contemplado desde el inicio del proyecto y se puede ver reflejado en la opción de menú de ``Comunicaciones'' de la ventana principal, la cual ofrece la opción de ``Otros''.

\subsection{Internacionalización}

Atendiendo a que el propósito de esta aplicación es educativo y se presenta en una universidad que tiene presencia a nivel internacional, resultaría provechoso añadir tantos idiomas como fueran posibles a la misma, ampliando así las facilidades dadas a los alumnos extranjeros.

Esta línea de trabajo se tuvo en cuenta desde el diseño de la ventana principal, dejando preparado un archivo de recursos con dos idiomas añadidos y enlazando todas las cadenas de texto de la aplicación con dicho archivo.

\subsection{TDD}

En mi opinión creo que resultaría producente realizar las futuras mejoras utilizando TDD (Test-Driven Development), consiguiendo de esta manera una mayor eficiencia y ahorrando tiempo a la larga en la resolución de futuros problemas que pudieran aparecer.

\subsection{Variables PID}

Actualmente la aplicación trabaja exclusivamente con variables normales, y no atiende a que éstas puedan ser de tipo PID, ni incluye la interfaz específica para ellas. La adición de esta mejora completaría el apartado relativo a las variables contenidas en un proyecto y abriría una abanico de funcionalidades nuevas a incluir a su vez.

\subsection{Estilos}

En esta versión la aplicación se muestra con los estilos predeterminados de una aplicación de Windows Forms. Podría resultar visualmente ventajoso trabajar sobre unos estilos con los que dotar a las ventanas, así como efectos y eventos al realizar acciones como el guardado de una variable o la selección de una imagen.

\subsection{Mejoras en las opciones de configuración}

Una de las mejoras que podría mejorar la aplicación notablemente sería la adición de una ventana en la que se pudieran configurar los parámetros de la comunicación, como por ejemplo, la velocidad de transmisión, paridad, etc.