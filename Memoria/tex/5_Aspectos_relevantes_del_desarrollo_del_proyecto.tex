\capitulo{5}{Aspectos relevantes del desarrollo del proyecto}

Este apartado pretende recoger los aspectos más importantes del desarrollo del proyecto, presentando los problemas que fueron aconteciendo y los caminos que se eligieron para solventarlos.

\section{Idea e inicio del proyecto}

La elección de este proyecto se vio motivada por varias razones. Por un lado, el interés personal por desarrollar una aplicación que se comunicara con un dispositivo hardware programable siempre había sido un asunto pendiente. Por otro lado, la posibilidad de ayudar a futuros alumnos de la Universidad de Burgos a tener una facilidad más con la que pudieran contar a la hora de llevar a cabo sus estudios, y evitando de este modo, que el proyecto cayera en el olvido.

Durante las primeras tomas de contacto con los tutores, se barajaron las posibilidades en las que el proyecto podría ser desarrollado, eligiendo en un principio trabajar en C++. Esta decisión se vio motivada porque la placa viene programada en ese lenguaje y por los conocimientos de los tutores en el mismo. Tras un pequeño periodo de formación en C++ y tras investigar las posibilidades que ofrecía C\# para cubrir las necesidades del proyecto, se replanteó el desarrollo del mismo, decidiendo cambiar a C\# y a la última versión de Visual Studio en la que me encontraba más cómodo.

Una vez definido el lenguaje de desarrollo se trataron los temas relacionados con el repositorio que se iba a utilizar, en este caso GitHub. También se comentaron las herramientas y entornos de desarrollo que se iban a utilizar (ZenHub, Visual Studio 2017), y la S con la que se iban a realizar reuniones para tratar los avances del proyecto, siendo periodos de dos semanas.

\section{Formación}

Desde que el tema fue elegido, hasta prácticamente la entrega de toda la documentación, ha sido una formación continua. Los distintos problemas y lagunas personales que se iban encontrando durante el desarrollo de la aplicación han servido como base para buscar el conocimiento necesario para poder finalizar con éxito dicha aplicación.

La mayoría de las herramientas utilizadas desde el inicio requirieron de una formación previa, ya fuera para ampliar conocimiento, como podría ser el caso de GitHub, o para crearlo, como en el caso de \LaTeX. Al igual que con las herramientas, también se realizó una investigación sobre el tema elegido, sumergiéndose así en el mundo del hardware NXP Freedom Freescale y las interfaces hombre-máquina.

Dentro de todo el conocimiento adquirido, cabe destacar todo aquel relacionado con C\#, tanto en la forma de estructurar un proyecto desde cero, como a la hora de programar. Por ejemplo, muchas líneas de código han sido ahorradas gracias a las expresiones Lambda, potenciando así la lógica y el pensamiento racional ante la programación básica con la que se contaba hasta el momento. Los manuales más consultados durante todo este tiempo han sido los proporcionados por Microsoft para dar luz sobre dudas a la hora de implementar clases o instrucciones no conocidas o no utilizadas, ergo se podría concluir que el proceso de formación más notable y provechoso en este proyecto ha sido en lo referente a la programación en C\# y la utilización de la herramienta Visual Studio 2017.

\section{•}





