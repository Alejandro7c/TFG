\capitulo{5}{Aspectos relevantes del desarrollo del proyecto}

Este apartado pretende recoger los aspectos más importantes del desarrollo del proyecto, presentando los problemas que fueron aconteciendo y los caminos que se eligieron para solventarlos.

\subsection{5.1. Inicio del proyecto}

La elección de este proyecto se vio motivada tanto por el interés personal por desarrollar una aplicación que se comunicara con un dispositivo como por la posibilidad de ayudar a la Universidad de Burgos y a sus alumnos a tener una facilidad más con la que pudieran contar a la hora de llevar a cabo sus estudios y que no fuera un proyecto que queda olvidado.

Durante las primeras tomas de contacto con los tutores, se barajaron las posibilidades en las que el proyecto podría ser desarrollado, eligiendo en un principio trabajar en C++. Esta decisión se vio motivada porque la placa viene programada en C++ y por los conocimientos de los tutores en este lenguaje. Tras un pequeño periodo de formación en C++ y tras investigar las posibilidades que ofrecía C\# para cubrir las necesidades del proyecto, se replanteó el desarrollo del mismo, decidiendo cambiar a C\# y a la última versión de Visual Studio en la que me encontraba más cómodo.

Una vez definido el lenguaje de desarrollo se trataron los temas relacionados con el repositorio que se iba a utilizar, en este caso GitHub, las herramientas y entornos de desarrollo que se iban a utilizar (ZenHub, Visual Studio 2017), y la periocidad con la que se iban a realizar reuniones para tratar los avances del proyecto, siendo periodos de dos semanas.

\subsection{5.1. Formación}

El proyecto requería de unos conocimientos de los que no se disponía originalmente, por lo que hubo que solventar estas cadencias.