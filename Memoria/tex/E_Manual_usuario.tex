\apendice{Documentación de usuario}

\section{Introducción}

En esta sección se presentará un manual de instalación y uso dirigido al usuario de la aplicación.

\section{Requisitos de usuarios}

El usuario necesitará del siguiente \textit{software} para ejecutar la aplicación:
\begin{itemize}
	\item \href{https://www.microsoft.com/es-es/download/details.aspx?id=49982}{.NET Framework 4.6.1.} Será necesario para tener todas las librerías necesarias para la ejecución de la aplicación. Si la versión es superior también valdría.
	\item \href{https://www.microsoft.com/es-es/sql-server/sql-server-downloads}{SQL Server 17.} Será utilizado para guardar los datos durante la ejecución de la aplicación. Es importante configurar el nombre de la conexión como \textit{``localhost$\slash$sqlexpress''}, sino habría que cambiar las cadenas de conexión dentro de la clase DB$\_$services.
	\item \href{https://github.com/FranBurgos/TFG/releases/tag/V_1.0}{Última \textit{release}.} Será necesario la descarga de la última versión del proyecto incluido en el archivo ``ReleaseV1.0.zip''.
\end{itemize}

\section{Instalación}

Para ejecutar la instalación, en primer lugar, deberemos haber instalado .NET Framework 4.6.1 y SQL Server 17 en nuestra máquina. Una vez hecho esto, sólo necesitaríamos descomprimir el archivo ``ReleaseV1.0.zip'' y ejecutar el archivo ``PlantaPiloto.exe''.

\section{Manual del usuario}

En esta sección se describe el uso de las funcionalidades de la aplicación.

\subsection{Crear configuración}

Una vez arrancada la aplicación, en el caso de que se quiera crear una nueva configuración para trabajar con ella, se deberá acceder a la opción de menú ``Configuración - Crear configuración'', la cual mostrará la ventana que se muestra en la Figura E.1.

\imagen{createConfigForm}{Formulario de creación de una configuración.}

En esta ventana se deberán rellenar los campos correspondientes a las propiedades del proyecto y añadir tantas variables como el usuario desee. Una vez hayamos creado la configuración deseada, se pinchará en el botón ``Aceptar'', el cual nos mostrará una ventana de diálogo donde se elegirá la ruta en la que se guardará el archivo con la configuración que acabamos de crear.

\subsection{Cargar configuración}

En el caso de que se disponga de un archivo de configuración ya creado y se quiera cargar directamente en la aplicación, se deberá ir a la opción de menú ``Configuración - Cargar configuración'', la cual abrirá una ventana de diálogo en la cual navegaremos por nuestro sistema de archivos hasta encontrar el archivo de configuración deseado.

\subsection{Modificar configuración}

Una vez que se tenga cargado un proyecto, ya sea recién creado o cargado desde archivo, la aplicación ofrece la posibilidad de modificar las propiedades del proyecto cargado (no permite la adición de nuevas variables). En la figura E.2 se puede apreciar cómo modificar el proyecto cargado.

\imagen{modifyConfigForm}{Formulario de modificación de una configuración.}

Cabe destacar que los cambios realizados sobre alguna de las variables del proyecto a modificar, no tendrán efecto hasta que se pulse el botón ``Guardar cambios en la variable''.

\subsection{Cambiar idioma}

Si el usuario tuviera la necesidad de cambiar de idioma la aplicación deberá dirigirse a la opción de menú ``Idioma'' y seleccionar el idioma que prefiera.

\subsection{Acceso a la ayuda}

En el caso de que se quiera acceder a la ayuda en línea que ofrece la aplicación, se podrá hacer desde cualquier ventana, ya sea a través de la opción de menú ``Ayuda - Ayuda'' en la ventana principal, o a través del botón de interrogación azul que se presenta en todas las ventanas.

\subsubsection{Manual de usuario}

Si en su defecto, el usuario prefiere consultar el manual de usuario que se incluye en la aplicación, no tendrá más que dirigirse a la opción de menú ``Ayuda - Manual de usuario'' para poder consultarlo.

\subsection{Abrir comunicación con puerto serie}

Para poder comenzar la comunicación con el puerto serie la aplicación requiere de:
\begin{itemize}
	\item Un proyecto cargado.
	\item Una conexión activa a través de un puerto serie. En el caso de que la placa haya sido conectada tras arrancar la aplicación, se podrá actualizar la lista de puertos disponibles a través del botón ``Actualizar''.
\end{itemize}

Teniendo estos requisitos cubiertos, el usuario sólo tendrá que pulsar el botón ``Inicio'' que se encuentra en el panel de ``Controles'' para comenzar la comunicación. Como se ve en la Figura E.3, una vez comenzada la comunicación los botones que antes estaban inactivos pasan a estar activos.

\imagen{mainFormOpened}{Formulario principal con la comunicación al puerto serie iniciada.}

\subsection{Cerrar comunicación con puerto serie}

Si, por el contrario, el usuario desea finalizar la comunicación con el puerto serie (es requisito que ésta esté iniciada), no tendrá más que pulsar en el botón ``Fin'' del panel de ``Controles''. Al hacer esto, el estado de los botones vuelve a cuando se acababa de cargar un proyecto.

\subsection{Selección de variables}

Una vez abierta la comunicación con el puerto serie, la aplicación ofrece una serie de funcionalidades para monitorizar los datos recibidos y almacenados en la base de datos. Para acceder a dichas funciones la aplicación incluye tres botones, ``Gráfica'', ``Variables'' y ``Archivo''. Estos botones tienen una parte de la ejecución común, que es la selección de las variables con las que se quiera trabajar. Como se ve en la Figura E.4, se seleccionarán las variables deseadas marcando con una palomita o chulito.

\imagen{varSelectionForm}{Formulario de selección de variables.}

Una vez seleccionadas las variables deseadas y según la funcionalidad buscada se presentan tres posibilidades:

\subsubsection{Graficado de los valores de las variables}

En el caso de haber pinchado en el botón ``Gráfica'', la ventana que se mostrará tras seleccionar las variables a mostrar, será la que aparece en la Figura E.5, en la cual se puede ver los valores que van teniendo las variables seleccionadas a lo largo del tiempo.

\imagen{chartForm}{Formulario de graficado de variables.}

Cabe destacar que la cantidad de valores mostrados puede ser modificada a través del cuadro de texto que se encuentra en la parte inferior izquierda y confirmando los cambios con el botón que hay a su lado.

\subsubsection{Mostrar los valores actuales de las variables}

Si el botón pulsado previo a la selección de variables corresponde con ``Variables'', la ventana que se mostrará contendrá los valores actualizados de las variables seleccionadas, como se muestra en la Figura E.6.

\imagen{varsForm}{Formulario de valores de variables seleccionadas.}

La diferencia entre este listado y el que se encuentra en la ventana principal es que este último solo muestra las variables de escritura, mientras que desde aquí se puede acceder a todas las variables del proyecto.

\subsubsection{Guardar valores de variables en un archivo}

En el caso de que el usuario precise guardar los valores que ciertas variables va teniendo a lo largo del tiempo, se pinchará el botón ``Archivo'', a través del cual, y una vez elegido las variables que se quieren guardar, la aplicación mostrará una ventana de diálogo donde se elegirá el destino del archivo y comenzará su guardado.

Una vez comenzado el guardado mencionado, el botón de la ventana principal cambiará a ``Detener guardado'', el cual pincharemos en el momento en el que se quiera detener el guardado de variables en archivo.
